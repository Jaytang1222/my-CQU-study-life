\documentclass[12pt,a4paper]{ctexart}
\usepackage{geometry}
\usepackage{pdfpages}
\usepackage{fancyhdr}
\usepackage{setspace}
\usepackage{titlesec}
\geometry{left=2cm,right=2cm,top=2.5cm,bottom=2.5cm}

% 页眉页脚设置
\pagestyle{fancy}
\fancyhf{}
\fancyfoot[C]{第 \thepage 页}
\renewcommand{\headrulewidth}{0pt}
\renewcommand{\footrulewidth}{0pt}

% 标题格式
\titleformat{\section}{\Large\bfseries}{\thesection}{1em}{}
\titleformat{\subsection}{\large\bfseries}{\thesubsection}{1em}{}

\title{电动汽车充电桩布局与推广模型研究(中文翻译版)}
\author{基于 2018 年 MCM/ICM D 题论文(Team \#80386)}
\date{}

\begin{document}
\maketitle
\tableofcontents
\newpage

%%%%%%%%%%%%%%%%%%%%%%%%%%%%%%%%%%%%%%%%%%%%%%%%%%%%%
% 每页:插入原PDF + 中文翻译内容
%%%%%%%%%%%%%%%%%%%%%%%%%%%%%%%%%%%%%%%%%%%%%%%%%%%%%

%----------------------- 第1页 -----------------------%
\includepdf[pages=1,scale=0.95,pagecommand={}]{341.pdf}

\section*{第1页中文译文}
随着猎鹰重型火箭的尾焰在大气中散去,特斯拉应当意识到——
将可靠的量产交付到消费者手中,比把一辆汽车送入太空更加困难。
本文旨在研究特斯拉如何制定发展规划,以满足客户需求,并让电动汽车进入千家万户。
针对任务1,我们对美国汽车数量进行回归分析,预测未来15年内所需的充电站数量为1,463,222个,
并探讨在城市、郊区与农村区域的合理分布……

\vspace{1cm}

%----------------------- 第2页 -----------------------%
\includepdf[pages=2,scale=0.95,pagecommand={}]{341.pdf}

\section*{第2页中文译文}
本页展示了论文目录结构和主要章节安排,包括引言、问题重述、模型假设、
模型参数、模型建立、灵敏度分析、优缺点以及为各国领导人撰写的建议信内容。
接下来章节将详细介绍模型设计与计算方法。

\vspace{1cm}

%----------------------- 第3页 -----------------------%
\includepdf[pages=3,scale=0.95,pagecommand={}]{341.pdf}

\section*{第3页中文译文}
\textbf{引言部分:}  
现代生活方式与生产方式的进步,使人类在享受便利的同时也面临能源短缺与生态失衡。
能源与环境危机在全球范围普遍存在,各国政府逐步重视电动汽车的发展,
以期缓解能源危机、推动社会经济可持续发展。
随着电动汽车的不断进步,合理规划充电站的选址与建设成为亟需解决的关键问题。

\vspace{1cm}

%----------------------- 第4页 -----------------------%
\includepdf[pages=4,scale=0.95,pagecommand={}]{341.pdf}

\section*{第4页中文译文}
\textbf{问题重述:}  
本文需解决的主要问题包括:
\begin{itemize}
    \item 探讨美国特斯拉充电站的类型与布局,判断是否能够在2030年前实现全面电动化;
    \item 建立韩国充电站布局模型,分析充电桩数量与分布的最优策略;
    \item 对不同国家条件(如人口与财富分布差异)建立分类系统;
    \item 探讨新交通方式(共享出行、无人驾驶、飞行汽车)及新技术对电动车产业的影响;
    \item 为国际能源会议提供针对不同国家的政策建议。
\end{itemize}

\vspace{1cm}

%----------------------- 第5页 -----------------------%
\includepdf[pages=5,scale=0.95,pagecommand={}]{341.pdf}

\section*{第5页中文译文}
\textbf{假设条件:}
\begin{enumerate}
    \item 每个充电站含8个充电桩;
    \item 当电动车普及至饱和状态时,与传统燃油车的使用量相等;
    \item 不考虑政治、军事及不可抗力影响;
    \item 模型基于现有数据与回归预测结果。
\end{enumerate}

\vspace{1cm}

%%%%%%%%%%%%%%%%%%%%%%%%%%%%%%%%%%%%%%%%%%%%%%%%%%%%%
% 提示:后续页可以用相同模板继续扩展至 page=22
%%%%%%%%%%%%%%%%%%%%%%%%%%%%%%%%%%%%%%%%%%%%%%%%%%%%%

% 例如批量插入(仅嵌入原PDF,不附中文):
% \includepdf[pages=6-22,scale=0.95,pagecommand={}]{341.pdf}

% 然后在需要的页后添加中文翻译段落:
% \section*{第6页中文译文}
% (在此填写对应翻译内容)

%%%%%%%%%%%%%%%%%%%%%%%%%%%%%%%%%%%%%%%%%%%%%%%%%%%%%

\end{document}
